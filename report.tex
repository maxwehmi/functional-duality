\documentclass[12pt,a4paper]{article}
\usepackage{etex,datetime,setspace,latexsym,amssymb,amsmath,amsthm}
\usepackage{fancybox,dialogue,float,wrapfig,enumerate,microtype}
\usepackage{verbatim,xcolor,multicol,titlesec,tabularx,mdframed}

\usepackage[utf8]{inputenc}
\usepackage[pdftex]{hyperref}
\usepackage[margin=2cm,bottom=3cm,footskip=15mm]{geometry}
\parindent0cm
\parskip0.5em

\usepackage{tikz}
\usetikzlibrary{arrows,trees,positioning,shapes,patterns}
\usetikzlibrary{intersections,calc,fpu,decorations.pathreplacing}

\usepackage[T1]{fontenc} % better fonts

% Haskell code listings in our own style
\usepackage{listings,color}
\definecolor{lightgrey}{gray}{0.35}
\definecolor{darkgrey}{gray}{0.20}
\definecolor{lightestyellow}{rgb}{1,1,0.92}
\definecolor{dkgreen}{rgb}{0,.2,0}
\definecolor{dkblue}{rgb}{0,0,.2}
\definecolor{dkyellow}{cmyk}{0,0,.7,.5}
\definecolor{lightgrey}{gray}{0.4}
\definecolor{gray}{gray}{0.50}
\lstset{
  language        = Haskell,
  basicstyle      = \scriptsize\ttfamily,
  keywordstyle    = \color{dkblue},     stringstyle     = \color{red},
  identifierstyle = \color{dkgreen},    commentstyle    = \color{gray},
  showspaces      = false,              showstringspaces= false,
  rulecolor       = \color{gray},       showtabs        = false,
  tabsize         = 8,                  breaklines      = true,
  xleftmargin     = 8pt,                xrightmargin    = 8pt,
  frame           = single,             stepnumber      = 1,
  aboveskip       = 2pt plus 1pt,
  belowskip       = 8pt plus 3pt
}
\lstnewenvironment{code}[0]{}{}

% only shown, not compiled:
\lstnewenvironment{showCode}[0]{\lstset{numbers=none}}{}

% only compiled, not shown:
\newcommand{\hide}[1]{}

% will the real phi please stand up
\renewcommand{\phi}{\varphi}

% load hyperref as late as possible for compatibility
\usepackage[pdftex]{hyperref}
\hypersetup{
  pdfborder = {0 0 0},
  breaklinks = true,
  linktoc = all,
}
\pdfinfoomitdate=1
\pdftrailerid{}
\pdfsuppressptexinfo15


\title{A Haskell implementation of translations between Distributive Lattices and Priestley Spaces}
\author{Giacomo de Antonelllis, \and Estel Koole, \and Marco de Mayda, \and  Edoardo Menorello, \and  Max Wehmeier}
\date{\today}
\hypersetup{pdfauthor={Giacomo de Antonelllis, Estel Koole, Marco de Mayda, Edoardo Menorello, Max Wehmeier}, pdftitle={A Haskell implementation of translations between Distributive Lattices and Priestley Spaces}}

\begin{document}

\maketitle


\begin{abstract}

The aim of this project is to provide a Haskell implementation for the Duality-theoretic translations between Distributive lattices and Priestley spaces.

In order to do so in Section 1 we define mappings. In section 2 we encode Partially ordered sets in Haskell, making use of the Data.set library. In section 2 we offer an implementation for Distributive lattices based on the on partially ordered set onstruction. In section 4 we define Priestley spaces in Haskell, making use of a prexisting Topology library. In section 5 we effectively implement translations between Priestley spaces and Distributive lattices.




% We give a toy example of a report in \emph{literate programming} style.
%The main advantage of this is that source code and documentation can
%be written and presented next to each other.
%We use the listings package to typeset Haskell source code nicely.

The repository of the project can be found at https://github.com/maxwehmi/functional-duality.
\end{abstract}

\vfill

\tableofcontents

\clearpage

% We include one file for each section. The ones containing code should
% be called something.lhs and also mentioned in the .cabal file.

\input{lib/Mapping.lhs}

\input{lib/Poset.lhs}

\input{lib/DL.lhs}

\input{lib/Priestley.lhs}


% I am leaving these in, so we remember that we still have to write them
\input{exec/Main.lhs}

\input{test/simpletests.lhs}


\section{Conclusion}\label{sec:Conclusion}

We have reached the following main objectives:

\begin{itemize}


    \item we have created a library for lattices and topological spaces. In particular we have provided an enconding for Distributive lattices and Priestley Spaces and their structural and their respective isomorphisms.  
    \item we have implemented in Haskell a translation between Distributive Lattices and Priestley Spaces. 
    \item we made the 
    \item we have constructed arbitrary instances of the structures we discussed in order to run \texttt{QuickCheck} tests on them.
    \item we have built the executable \texttt{main} to serve as a command line interface.  
    \item we have constructed a parser to make the \texttt{main} program more accessible.   
    \item when running the \texttt{main} executable file, among the other functionalities available, the user will be able to visualise the structures that he wishes to imput, he will be able to retreive the dual of such structures, and he will be able to check the isomorphism between a structure and its dual's dual. 
\end{itemize}

The span of the project is limited to finite structures, but the objects have been implemented in such a way that their Haskell types allow the expansion to infinite structures. This project can be seen as a foundational starting point for such an endeavour. 

In conclusion we hope that the reader will find such an implementation of interest and, most importantly, that he will be able to take advantage of it to foster his studies in both the theory and applications of this area of mathematics. 

\addcontentsline{toc}{section}{Bibliography}
\bibliographystyle{alpha}
\bibliography{references.bib}

\end{document}

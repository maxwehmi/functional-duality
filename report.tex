\documentclass[12pt,a4paper]{article}
\input{latexmacros.tex}

\title{A Haskell implementation of translations between Distributive Lattices and Priestley Spaces}
\author{Giacomo de Antonelllis, Estel Koole, Marco de Mayda, Edoardo Menorello, Max Wehmeier}
\date{\today}
\hypersetup{pdfauthor={Giacomo de Antonelllis, Estel Koole, Marco de Mayda, Edoardo Menorello, Max Wehmeier}, pdftitle={A Haskell implementation of translations between Distributive Lattices and Priestley Spaces}}

\begin{document}

\maketitle

% I am leaving this in, so we remember that we still have to write them
\begin{abstract}
We give a toy example of a report in \emph{literate programming} style.
The main advantage of this is that source code and documentation can
be written and presented next to each other.
We use the listings package to typeset Haskell source code nicely.
\end{abstract}

\vfill

\tableofcontents

\clearpage

% We include one file for each section. The ones containing code should
% be called something.lhs and also mentioned in the .cabal file.

\input{lib/Mapping.lhs}

\input{lib/Poset.lhs}

\input{lib/Priestley.lhs}

\input{lib/DL.lhs}

% I am leaving these in, so we remember that we still have to write them
\input{exec/Main.lhs}

\input{test/simpletests.lhs}


\section{Conclusion}\label{sec:Conclusion}

We have reached the following objectives:

\begin{itemize}


    \item we have created encodings for posets, lattices, distributive lattices, Priestley spaces, 
\end{itemize}

\addcontentsline{toc}{section}{Bibliography}
\bibliographystyle{alpha}
\bibliography{references.bib}

\end{document}
